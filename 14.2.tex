本节我们讨论几个经典的积分公式在场论语言下的具体形式,以及由此可以得到的物理结论.

\mysubsection{向量形式的积分公式}

\mysubsubsection{几个常用记号的回顾}

\begin{itemize}
    \item 设 $D\subset\RR^n$ 为开集,$A:D\to\RR^n$ 为向量场. 设 $\gamma$ 为 $D$ 中的一条光滑定向曲线,其在每点 $x$ 处的单位切向量为 $e(x)$. 设 $\dd s$ 为 $\gamma$ 上的长度元,则有
$$
\begin{aligned}
    \omega_A^1&=\sum_{i=1}^nA_i\dd x_i\\
    \implies\omega_A^1|_\gamma&=\sum_{i=1}^nA_ie_i(x)\dd s=\inner{A,e}\dd s\\
    &=\inner{A,\dd\mathbf{s}}=A\cdot\dd\mathbf{s}
\end{aligned}
$$
    
    其中 $\dd\mathbf{s}\triangleq e\dd s$ 为向量长度元.

    \item 设 $D\subset\RR^3$ 为开集,$B:D\to\RR^3$ 为向量场.
    
    设 $S\subset D$ 为二维光滑定向曲面. 其定向由单位法向量场 $n(x)$ 指定. 设 $\dd\sigma$ 为 $S$ 上的面积元,则
$$
\begin{aligned}
    \omega_B^2&=B_1\dd y\wedge\dd z+B_2\dd z\wedge\dd x+B_3\dd x\wedge\dd y\\
    \implies\omega_B^2|_S&=B_1n_1(x)\dd\sigma+B_2n_2(x)\dd\sigma+B_3n_3\dd\sigma\\
    &=\inner{B,\dd\bm{\sigma}}=B\cdot\dd\bm{\sigma}
\end{aligned}
$$

    其中 $\dd\bm{\sigma}\triangleq n\dd\sigma$ 为向量面积元.
\end{itemize}

\mysubsubsection{Newton-Leibniz 公式}

\begin{theorem}  
    设 $D\subset\RR^n$ 为开集,$f\in C^{(1)}(D)$. 设 $\gamma\in D$ 为光滑曲线且有参数化 $\varphi:[a,b]\to\gamma$. 则
$$
\int_\gamma\nabla f\cdot\dd\mathbf{s}=f(\varphi(b))-f(\varphi(a))=f\biggr |_{\partial\gamma}
$$
\end{theorem}
\begin{proof}
$$
\begin{aligned}
    &\int_\gamma\nabla f\cdot\dd\mathbf{s}=\int_\gamma\sum_{i=1}^n\pard{f}{x_i}(x)\dd x_i\\
    =&\int_a^b\sum_{i=1}^n\pard{f}{x_i}(\varphi(t))\varphi_i'(t)\dd t=\int_a^b(f\circ\varphi(t))'\dd t\\
    =&f\circ\varphi(b)-f\circ\varphi(a)
\end{aligned}
$$
\end{proof}

\textcolor{mydarkblue}{物理意义:}$f$ 的梯度沿 $\gamma$ 所做的功仅与 $f$ 在起点与终点的值有关. 进一步,仅与两个水平集的水平值有关.

\img{0.7}{14.2.1.png}

如图所示,则有 $\displaystyle\int_{\gamma_1}\nabla f\cdot\dd\mathbf{s}=\int_{\gamma_2}\nabla f\cdot\dd\mathbf{s}$.

\begin{example}
    我们之前讨论过场 $F(r)=f(\abs{r})\cdot r$.

    此时,$F$ 可以写成某个函数 $G$ 的梯度.
\end{example}

\mysubsubsection{经典 Stokes 公式}

\begin{theorem}
    设 $D\subset\RR^3$ 为开集,$A:D\to\RR^3$ 光滑. 设 $S\subset D$ 为二维光滑带边定向曲面,$\overline{S}$ 紧.
    
    则 Stokes 公式为
$$
\int_{\partial S}\omega_A^1=\int_S\dd\omega_A^1=\int_S\omega_{\nabla\times A}^2
$$

    进而可以写成
$$
\oint_{\partial S}A\cdot\dd\mathbf{s}=\int_S\nabla\times A\cdot\dd\bm{\sigma}
$$
\end{theorem}

\textcolor{mydarkblue}{物理意义:}向量场 $A$ 沿曲线 $\partial S$ 的环量 $=$ 其旋度 $\nabla\times A$ 穿过曲面 $S$ 的通量.

\img{0.5}{14.2.2.png}

\begin{example}
    Faraday 定律
$$
\oint_{\partial S} E\cdot\dd\mathbf{s}=-\int_S\pard{B}{t}\cdot\dd\bm{\sigma}
$$

    对应于 Maxwell 方程组中的 $\displaystyle\nabla\times E=-\pard{B}{t}$.
\end{example}

\begin{example}
    Ampère 定律
$$
\oint_{\partial S}B\cdot\dd\mathbf{s}=\frac{1}{\eps_0c^2}\int_S j\cdot\dd\bm{\sigma}
$$

    对应于 Maxwell 方程组中的 $\displaystyle\nabla\times B=\frac{j}{\eps_0c^2}+\frac{1}{c^2}\pard{E}{t}$.

    因为在静电场中有 $\displaystyle\pard{E}{t}=0$.
\end{example}

\mysubsubsection{Gauss-Ostrogradskii 公式}

\begin{theorem}
    设 $D\subset\RR^3$ 为开集,$A:D\to\RR^3$ 光滑. 设 $G\subset D$ 为区域,$\overline G$ 紧,$\partial G$ 分片光滑.

    则 G-O 公式为
$$
\int_{\partial G}\omega_A^2=\int_G\dd\omega_A^2=\int_G\omega_{\nabla\cdot A}^3
$$

    进而可以写成
$$
\int_{\partial G}A\cdot\dd\bm{\sigma}=\int_G\nabla\cdot A\dd V
$$

    其中 $\dd V=\dd x\wedge\dd y\wedge\dd z$ 为体积元.
\end{theorem}

\textcolor{mydarkblue}{物理意义:}$A$ 穿过区域 $G$ 边界 $\partial G$ 的通量 $=$ 其散度 $\nabla\cdot A$ 在区域 $G$ 上的积分.

\mysubsection{梯度、散度、旋度的物理意义}

\mysubsubsection{梯度}

设 $D\subset\RR^3$ 为开集,$f:D\to\RR$ 光滑.

则 $\nabla f(p)$ 代表了在 $p$ 处 $f$ 增长最快的方向.

记 $c=f(p)$. 设 $S_c=\set{p'\in D|f(p')=c}$. 则 $\nabla f(p)\perp T_pS_c$.

\img{0.6}{14.2.3.png}

设 $\xi\in\RR^3$,则 $f$ 沿 $\xi$ 的方向导数为
$$
D_\xi f(p)=\nabla f(p)\cdot\xi=\inner{\nabla f(p),\xi}
$$

\mysubsubsection{散度}

设 $A$ 是一个向量场. 在实际应用中,$A$ 通常是由一些源产生的. 例如,电荷可以产生电场,热源可以产生热流,水源可以产生水流,等等. 我们来看 G-O 公式能告诉我们什么.

设 $x\in D$. 取 $r>0$ 使得 $B(x,r)\subset D$. 我们将 G-O 公式应用于 $B(x,r)$,得到
$$
\int_{B(x,r)}\nabla\cdot A\dd V=\int_{\partial B(x,r)}A\cdot\dd\bm{\sigma}
$$

应用积分中值定理可得
$$
\exists \widetilde{x}\in B(x,r),\nabla\cdot A(\widetilde{x})V(B(x,r))=\int_{B(x,r)}\nabla\cdot A\dd V=\int_{\partial B(x,r)}A\cdot\dd\bm{\sigma}
$$

若将 $A$ 解释为流速,则右边为单位时间内穿过 $\partial B(x,r)$ 的流量,或者等价地说,流出 $B(x,r)$ 的体积. 从而
$$
\nabla\cdot A(\widetilde{x})=\frac{1}{V(B(x,r))}\int_{\partial B(x,r)}A\cdot\dd\bm{\sigma}
$$

为单位体积内流出体积的强度.

令 $r\to 0$. 由 $A$ 光滑可得
$$
\nabla\cdot A(x)=\lim_{r\to 0}\frac{1}{V(B(x,r))}\int_{\partial B(x,r)}A\cdot\dd\bm{\sigma}
$$

从而 $\nabla\cdot A(x)$ 测量了 $x$ 处流的源的分布密度.

\begin{example}
    若 $\nabla\cdot A=0$,则说明 $A$ 是一个无源场.

    从而对任意的封闭曲面 $S\subset D$ 总有 $\displaystyle\int_SA\cdot\dd\bm{\sigma}=0$.
\end{example}

\begin{example}[ Gauss 定律]
    由 Maxwell 方程组以及 G-O 公式有:设 $G\subset\RR^3$ 为区域,则
$$
\int_{\partial G}E\cdot\dd\bm{\sigma}=\int_G\nabla\cdot E\dd V=\int_G\frac{\rho}{\eps_0}\dd V=\frac{q}{\eps_0}
$$

    其中 $q$ 为区域 $G$ 内的总电荷. 上式即为 Gauss 定律.
\end{example}

\begin{example}[ Coulomb 定律]
    我们利用 Gauss 定律来推导 Coulomb 定律.

    在原点 $O$ 处放一个点电荷 $q$. 首先由对称性,空间电场是球对称的,故 $r\in\RR^3$ 处的电场应有方向 $r$,且在半径为 $\abs{r}$ 的球面上大小相同. 设其为 $E$. 则由 Gauss 定律有
$$
\frac{q}{\eps_0}=\int_{\partial B(0,\abs{r})}E\cdot\dd\bm{\sigma}=4\pi|r|^2|E|
$$

    由此得到
$$
E(r)=\frac{q}{4\pi\eps_0}\frac{1}{\abs{r}^2}\frac{r}{|r|}
$$

    我们再对此结论进行进一步的观察:

    直接计算可以得到 $\nabla\times E(r)=0,\forall r\ne 0$.

    这与 $\nabla\times E$ 的物理意义是吻合的. 因为在 $r\ne 0$ 处以及周围没有电荷,从而电荷密度为 $0$.

    另一方面,我们来计算 $\nabla\cdot E$ 在 $0$ 处的取值. 由 Gauss 定律有
$$
\nabla\cdot E(0)=\lim_{r\to 0}\frac{1}{V(B(0,r))}\int_{B(0,r)}E\cdot\dd\bm{\sigma}=\lim_{r\to 0}\frac{3q}{4\eps_0\pi r^3}=+\infty
$$

    即在 $0$ 处的电荷密度为无穷大. 综上
$$
\nabla\cdot E(r)=\begin{cases}
    0 & (r\ne 0)\\
    +\infty & (r=0)
\end{cases}
$$

    二者从物理意义上讲都十分的清楚,但在数学上却造成了一定的麻烦. 因为我们有一个在除了 $0$ 点外均为 $0$ 的函数,但在 $0$ 点处的值为 $+\infty$,且其在 $\RR^3$ 上的积分为一个常值 $\dfrac{q}{\eps_0}$.

    在后续的课程中会对这样的函数做出严格的定义,其被称为 Dirac 函数 $\delta$. 其物理意义即为质点的密度函数.(在测度论中,其为 Dirac 测度)在分布理论中,其被认为是一个广义函数.
\end{example}

\begin{example}
    Maxwell 方程组断言 $\nabla\cdot B=0$.

    这表明磁场 $B$ 是一个无源场. 这与不存在“磁单极子”的事实是吻合的.
\end{example}

\mysubsubsection{旋度}

旋度的物理意义相对来说没有那么直观. 我们首先来看一个例子,其表明旋度的确与旋转密切相关.

\begin{example}
    考虑空间中绕 $z$ 轴的匀速转动. 设角速度大小为 $\omega$. 用向量场来刻画,即为速度场
$$
V:\RR^3\to\RR^3,(x,y,z)\mapsto(-\omega y,\omega x,0)
$$

    \img{0.4}{14.2.4.png}

    此时我们有
$$
\nabla\times V(x,y,z)=\begin{vmatrix}
    \hat{\imath} & \hat{\jmath} & \hat{k}\\
    \pard{}{x} & \pard{}{y} & \pard{}{z}\\
    -\omega y & \omega x & 0    
\end{vmatrix}=(0,0,2\omega)
$$

    即速度场 $V$ 的旋度给出了旋转轴及其方向(向上)以及角速度大小(两倍于角速度).
\end{example}

接下来我们讨论旋度更为深刻一些的物理意义. 其仅在借助于 Stokes 公式之后才逐渐变得清晰.

设 $D\subset\RR^3$ 为开区域. 其上定义了某个流的速度场 $V:D\to\RR^3$ 光滑. 取定 $p\in D$.

取定一个单位向量 $n$. 以及与 $n$ 垂直的一个以 $p$ 为圆心,以 $r$ 为半径的小圆盘 $S(r,n)$.

\img{0.2}{14.2.5.png}

由 Stokes 公式有
$$
\int_{\partial S}V\cdot\dd\mathbf{s}=\int_{S(r,n)}\nabla\times V\cdot\dd\bm{\sigma}
$$

即 $V$ 沿圆盘边界的环量 $=$ 其旋度 $\nabla\times V$ 穿过圆盘的通量. 由介值定理有
$$
\begin{aligned}
    \exists p'\in S(r,n),\nabla\times V(p')\cdot n\cdot\pi r^2&=\int_{S(r,n)}(\nabla\times V\cdot n)\dd\sigma\\
    &=\int_{S(r,n)}\nabla\times V\cdot\dd\bm{\sigma}\\
    &=\int_{\partial S(r,n)}V\cdot\dd\mathbf{s}    
\end{aligned}
$$

从而
$$
\nabla\times V(p)\cdot n=\lim_{r\to 0}\frac{1}{\pi r^2}\int_{\partial S(r,n)}V\cdot\dd\mathbf{s}
$$

我们暂且将右边的量称为 $V$ 沿 $n$ 方向的环量密度.

从而上式告诉我们:当 $n$ 沿着 $\nabla\times V(p)$ 的方向时,其密度达到最大值,且其密度的最大值恰为 $\abs{\nabla\times V(p)}$.

另一方面,设由 $V$ 导致的运动在 $p$ 点的旋转部分的旋转轴为 $\bm{\omega}$,大小为 $\abs{\bm{\omega}}$.

则由环量的几何意义知:当 $n$ 与 $\bm{\omega}$ 同向时,得到的环量密度达到最大.

\img{0.4}{14.2.6.png}

从而有 $\nabla\times V(p)=\bm{\omega}$.

\mysubsection{其它的积分公式}

以下我们给出一些进一步的积分公式. 它们均可以从经典的积分公式中得到,但在不同的场合有不同的用处.

\mysubsubsection{向量版本的 G-O 公式}

\begin{property}
    \begin{enumerate}
        \item\label{G-O1}
$$
\int_{\partial D}\dd\bm{\sigma}\cdot B=\int_D\nabla\cdot B\dd V
$$

        \item\label{G-O2}
$$
\int_{\partial D}\dd\bm{\sigma}\times B=\int_D\nabla\times B\dd V
$$

        \item\label{G-O3}
$$
\int_{\partial D}\dd\sigma f=\int_D\nabla f\dd V
$$
    \end{enumerate}
\end{property}

其中 \ref{G-O1} 为经典的 G-O 公式,\ref{G-O2} 与 $\ref{G-O3}$ 可由将 \ref{G-O1} 应用到每个分量得到.

有了这个公式,我们可以统一地将散度、旋度与梯度表达为某种形式的密度.
$$
\begin{aligned}
    \nabla\cdot B(x)&=\lim_{r\to 0}\frac{1}{V(B(x,r))}\int_{\partial B(x,r)}\dd\bm{\sigma}\cdot B\\
    \nabla\times B(x)&=\lim_{r\to 0}\frac{1}{V(B(x,r))}\int_{\partial B(x,r)}\dd\bm{\sigma}\times B\\
    \nabla f(x)&=\lim_{r\to 0}\frac{1}{V(B(x,r))}\int_{\partial B(x,r)}\dd\sigma\cdot f\\
\end{aligned}
$$

\mysubsubsection{向量版本的 Stokes 公式}

\begin{property}
    \begin{enumerate}
        \item 
$$
\int_{\partial S}\dd\bm{s}\cdot A=\int_S\dd\bm{\sigma}\cdot(\nabla\times A)
$$

        \item 
$$
\int_{\partial S}\dd\bm{s}\times A=\int_S(\dd\bm{\sigma}\times \nabla)\times A
$$

        \item 
$$
\int_{\partial S}\dd\bm{s}\cdot f=\int_S\dd\bm{\sigma}\times\nabla f
$$
    \end{enumerate}
\end{property}

\mysubsubsection{Green 公式}

\begin{property}
    \begin{enumerate}
        \item 
$$
\int_{\partial D}(g\nabla f)\cdot\dd\bm{\sigma}=\int_D(\nabla g\cdot\nabla f+g\Delta f)\dd V
$$

        \item 
$$
\int_{\partial D}(g\nabla f-f\nabla g)\cdot\dd\bm{\sigma}=\int_D(g\Delta f-f\Delta g)\dd V
$$
    \end{enumerate}
\end{property}

\begin{inference}
    \begin{enumerate}
        \item 
$$
\int_{\partial D}f\nabla f\cdot\dd\bm{\sigma}=\int_D((\nabla f)^2+f\Delta f)\dd V
$$

        \item 
$$
\int_{\partial D}\nabla f\cdot\dd\bm{\sigma}=\int_D\Delta f\dd V
$$
    \end{enumerate}
\end{inference}