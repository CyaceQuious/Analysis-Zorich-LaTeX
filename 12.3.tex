在实际应用中,我们经常面对有边界的曲面. 本节我们来讨论它们的严格定义以及定向.

\img{0.6}{12.3.1.png}

\mysubsection{带边曲面的定义}

正如 $\RR^k$ 是 $k$ 维曲面的标准局部模型,带边曲面也有一个标准的局部模型. 令
$$
H^k=\set{t\in\RR^k|t_1\le 0}
$$

\img{0.4}{12.3.2.png}

则
$$
\begin{aligned}
\partial H^k&=\set{t\in\RR^k|t_1=0}\simeq\RR^{k-1}\\
\mathring{H^k}&=\set{t\in\RR^k|t_1<0}\simeq\RR^k
\end{aligned}
$$

从这种角度理解,$H$ 是由 $k$ 维曲面 $\mathring{H^k}$ “粘上”一个 $k-1$ 维曲面 $\partial H^k$ 得到的.

\begin{definition}
设 $S\subset\RR^n$ ,称 $S$ 是一个带边 $k$ 维曲面,若 $\forall x\in S$ 存在 $x$ 的邻域 $U(x)\subset S$ 使得:

$U(x)$ 要么与 $\RR^k$ 同胚,要么与 $H^k$ 同胚.
\end{definition}

直观的说:$S$ 的每个小局部看起来要么像 $\RR^k$ ,要么像 $H^k$.

\begin{definition}
设 $S\subset\RR^n$ 为 $k$ 维带边曲面.

若对 $x\in S$ 存在 $x$ 的邻域 $U(x)\subset S$ 以及 $\varphi:H^k\to U(x)$ 为同胚,且满足 $x\in\varphi(\partial H^k)$ ,则称 $x$ 是 $S$ 的边界点.

用 $\partial S$ 记 $S$ 所有边界点的集合.
\end{definition}

\begin{hint}
\begin{enumerate}
    \item 我们用局部图来定义边界点,从而 $\partial S$ 的定义有可能依赖于局部图的选取. 但事实上我们有:
    
    \begin{property}
        $\partial S$ 不依赖于局部图的选取.
    \end{property}
    \begin{proof}
        利用以下定理:
    \end{proof}

    \begin{theorem}[Brouwer 定理]
        设 $A,B\subset\RR^k,\varphi:A\to B$ 为同胚,则 $\varphi(\partial A)=\partial B,\varphi(\mathring{A})=\mathring{B}$.
    \end{theorem}

    \item 在我们的定义中包含了 $\partial S=\varnothing$ 的情形,从而包含了之前曲面的定义. 以后提到带边曲面 $S$ 时,总假设 $\partial S\ne\varnothing$.
    
    \item 当 $k=1$ 时,注意到 $H^1=\set{t\in\RR|t\le 0}\implies \partial H^1=\set{0}$ 为单点集. 今后我们使用记号 $\RR^0$ 来表示单点集,并约定 $\partial\RR^0=\varnothing$.
\end{enumerate}
\end{hint}

类似于曲面情形,我们对带边曲面也可以定义其局部图与图册.

且可以证明:只需至多可数个图的有效域即可覆盖 $S$.

\begin{definition}
设 $\set{\varphi_i:\RR^k\to U_i}\cup\set{\psi_j:H^k\to V_j}$ 是带边曲面 $S$ 的一个图册.

若 $\forall i,j$ 均有 $\varphi_i\in C^{(m)}(\RR^k),\psi_j\in C^{(m)}(H^k)$ 且 $\mathrm{r}(\varphi_i')\equiv k\equiv\mathrm{r}(\psi_j')$ ,则称 $S$ 为 $m$ 阶光滑曲面.
\end{definition}

\begin{property}
若 $S$ 为 $k$ 维带边曲面,$\set{\varphi_i:\RR^k\to U_i}\cup\set{\psi_j:H^k\to V_j}$ 是 $S$ 的一个图册.

则 $\partial S$ 是 $k-1$ 维曲面,且 $\set{\psi_j|_{\partial H^k}}$ 是 $\partial S$ 的一个图册.

若 $S$ 为 $m$ 阶光滑,则 $\partial S$ 也为 $m$ 阶光滑.
\end{property}
\begin{proof}
    由定义.
\end{proof}

\begin{example}
    $\overline{B}^n=\set{x\in\RR^n:\abs{x}\le 1}$ 为 $C^{\infty}$ 光滑 $n$ 维带边曲面,其边界为 $\partial\overline{B}^n=S^{n-1}$.
\end{example}

\begin{example}
    $\overline{I}^2=\set{(x,y)\in\RR^2:\abs{x},\abs{y}\le 1}$ 是 $2$ 维带边曲面.
    
    其边界为 $\partial\overline{I}^2=\set{(x,y)\in\RR^2:\max\set{\abs{x},\abs{y}}=1}$.

    但 $\overline{I}^2$ 不是光滑带边曲面.

    \img{0.4}{12.3.3.png}

    而 $\overline{I}^2\setminus\set{a,b,c,d}$ 为 $2$ 维光滑带边曲面.

    其边界 $(a,b)\cup(b,c)\cup(c,d)\cup(d,a)$ 为四段开区间的并,也是光滑 $1$ 维曲面,但不连通.
\end{example}

\begin{example}
    将闭长方体以如下方式粘连得到光滑柱面 $C$.

    \img{0.6}{12.3.4.png}

    其边界为 $S_1\cup S_2$ ,为 $1$ 维光滑曲线. 注意到 $C$ 与如图的 $S$ 同胚.
\end{example}

\begin{example}
    将闭长方体以如下方式粘连得到 Möbius 带.

    \img{0.5}{12.3.5.png}

    只要粘连方式足够规则,则 $M$ 是一个光滑 $2$ 维带边曲面.
    
    其边界是一条封闭的空间曲线,而非两条分离的圆周.
\end{example}

\mysubsection{带边曲面及其边界的定向}

设 $S\subset\RR^n$ 是一个 $k$ 维光滑带边曲面,我们可以用与无边曲面相同的方式对其定义可定向性,即要求 $S$ 有一个定向图册.

\begin{property}
    若 $S$ 为 $k$ 维光滑带边曲面,且可定向.

    设 $\mathscr{A}=\set{\varphi_i:\RR^k\to U_i}\cup\set{\psi_j:H^k\to V_j}$ 是 $S$ 的一个定向图册.

    则 $k-1$ 维光滑曲面 $\partial S$ 也可定向,且 $\set{\psi_j|_{\partial H^k}}$ 是 $S$ 的一个定向图册.
\end{property}
\begin{proof}
    令 $\widetilde\psi_j=\psi_j|_{\partial H^k},\widetilde{V}_j=V_j\cap\partial S$.

    我们只需证明 $\widetilde\psi_i$ 与 $\widetilde\psi_j$ 相容,为此设 $\widetilde{V}_i\cap\widetilde{V}_j\ne\varnothing$.

    \img{0.8}{12.3.6.png}

    任取 $x\in\widetilde{V}_i\cap\widetilde{V}_j$. 通过适当的平移,不妨设 $\psi_i(0)=\psi_j(0)=x$.

    令 $\Psi=\psi_j^{-1}\circ\psi_i$. 则由 $\psi_i$ 与 $\psi_j$ 相容知 $\det J_\Psi(0)>0$.

    令 $\widetilde\Psi=\widetilde\psi_j^{-1}\circ\widetilde\psi_i$. 更确切的,$\widetilde\Psi(t_2,\cdots,t_k)\triangleq\widetilde\psi_j^{-1}\circ\widetilde\psi_i(0,t_2,\cdots,t_k)=\Psi(0,t_2,\cdots,t_k)$.

    则我们只需证 $\det J_{\widetilde\Psi}(0)>0$.

    由 $\Psi$ 的定义知 $\Psi(0,t_2,\cdots,t_k)\subset\partial H^k$.

    从而 $\Psi_1(0,t_2,\cdots,t_k)\equiv 0$. 则有
$$
J_\Psi(0)=\left[\begin{array}{c|ccc}
    \partial_1\Psi(0) & 0 & \cdots & 0\\
    \hline
    * & \partial_2\widetilde\Psi_2(0) & \cdots & \partial_k\widetilde\Psi_2(0)\\
    \vdots & \vdots & \ddots & \vdots\\
    * & \partial_2\widetilde\Psi_k(0) & \cdots & \partial_k\widetilde\Psi_k(0)
\end{array}\right]=\begin{bmatrix}
    \partial_1\Psi(0) & 0\\
    * & J_{\widetilde\Psi}(0)
\end{bmatrix}
$$

    故 $\det J_\Psi(0)=\partial_1\Psi(0)\cdot\det J_{\widetilde\Psi}(0)>0$.

    由定义知当 $t_1<0$ 时 $\Psi_1(t_1,0,\cdots,0)<0$ ,从而 $\partial_1\Psi(0)>0$.

    从而 $\det J_{\widetilde\Psi}(0)>0$.
\end{proof}

我们来解释一下其中的直观. 首先我们取定 $\RR^k$ 的标准定向 $(e_1,\cdots,e_k)$ ,并取 $(e_2,\cdots,e_k)$ 为 $\partial H^k$ 的定向.

\img{0.4}{12.3.7.png}

直观来说,第一个向量 $e_1$ 指向 $H^k$ 的外部,此时我们称 $(e_2,\cdots,e_k)$ 是 $\partial H^k$ 上的与 $H^k$ 上的定向 $(e_1,\cdots,e_n)$ 相容的定向.

当 $k=1$ 时,$\partial H^1$ 为单点集. 我们规定 $\partial H^1$ 的定向为 $+$ ,并称其是与 $\mathring{H}^1$ 上的定向 $(e_1)$ 相容的定向.

\img{0.6}{12.3.8.png}

现在我们回到直观的理解.

\img{0.8}{12.3.9.png}

由 $\psi_i,\psi_j$ 的定义知 $\psi'_i(0)$ 与 $\psi'_j(0)$ 将 $e_1$ 映射到 $T_xS$ ,且指向曲面外($T_xS$ 的同侧).

而 $\psi'_i(0)$ 与 $\psi'_j(0)$ 同时将 $(e_1,\cdots,e_k)$ 映射为 $T_xS$ 中的标架,且相容,从而限制在 $T_x\partial S$ 上也相容.

由此性质可知:

\begin{definition}
    设 $S$ 为 $k$ 维带边光滑曲面,且可定向.

    设 $\mathscr{A}(S)=\set{\varphi_i:\RR^k\to U_i}\cup\set{\psi_j:H^k\to V_j}$ 是 $S$ 的一个定向图册.

    则 $\mathscr{A}(\partial S)=\set{\psi_j|_{\partial H^k}}$ 是 $\partial S$ 的一个定向图册.

    此时我们称由 $\mathscr{A}(\partial S)$ 指定的 $\partial S$ 的定向与由 $\mathscr{A}(S)$ 指定的 $S$ 的定向相容.
\end{definition}

\begin{hint}
\begin{enumerate}
    \item 在实际应用中,为了确定在 $S$ 和 $\partial S$ 上的相容定向,我们可以采取如下的方式:
    
    任取 $x_0\in\partial S$ ,先取 $\xi_1\in T_{x_0}S$ 使得 $\xi_1\perp T_{x_0}\partial S$ 且指向曲面的外部.

    再将 $\xi_1$ 扩充成 $T_{x_0}S$ 上的标架,使得 $\xi_2,\cdots,\xi_k\in T_{x_0}\partial S$.

    则此时 $(\xi_1,\cdots,\xi_k)$ 在 $S$ 上指定的定向与 $(\xi_2,\cdots,\xi_k)$ 在 $\partial S$ 上指定的定向相容.

    \img{0.4}{12.3.10.png}

    \item 我们来考虑一个很有启发性的例子.
    
    令 $H^k_-\triangleq H^k$ ,与 $H^k_+\triangleq\set{t\in\RR^k|t_1\ge 0}$ 均为 $k$ 维带边曲面,且设它们都继承了 $\RR^k$ 的标准定向 $(e_1,\cdots,e_k)$.

    \img{0.4}{12.3.11.png}

    则它们在公共边界 $\Gamma=\set{t\in\RR^k|t_1=0}$ 上诱导的相容定向方向恰好相反(不同).

    更一般的,若一个 $k$ 维可定向的曲面被一个 $k-1$ 维光滑曲面割开,则在边界上两个部分也会诱导相反的定向.

    \img{0.5}{12.3.12.png}
\end{enumerate}
\end{hint}

以上的观察可以让我们将光滑(可定向)曲面的定义推广到分片光滑(可定向)曲面,在实际应用中会很方便.

\begin{definition}
    我们递归地定义分片光滑曲面如下:

    \begin{itemize}
        \item 单点集 $\set{x}\subset\RR^n$ 是 $0$ 维分片光滑曲面,且任意阶光滑.
        
        \item 设 $S$ 为 $k$ 维曲面(不一定光滑).
        
        称 $S$ 为 $k$ 维分片光滑曲面,若在去掉至多可数个 $\le k-1$ 维分片光滑曲面后,$S$ 是若干个 $k$ 维光滑曲面的(不交)并 $\bigcup_i S_i$.
    \end{itemize}
\end{definition}

\begin{hint}
    递归地分解一个 $k$ 维分片光滑曲面,可以得到:
    
    若 $S$ 为 $k$ 维分片光滑曲面,则
$$
S=\bigsqcup_iS_i
$$

    其中每个 $S_i$ 均为 $k_i$ 维光滑曲面,其中 $k_i\le k$.
\end{hint}

\begin{example}
    $\overline{I}^2$ 本身不是光滑曲面,但 $\overline{I}^2$ 去掉边界后是 $2$ 维光滑曲面.

    从而 $\overline{I}^2$ 是 $2$ 维分片光滑曲面.
\end{example}

接下来我们讨论分片光滑曲面的定向. 我们从 $0$ 维开始.

我们在 $\RR^0$ 上指定两种定向 $(\RR^0,+)$ 和 $(\RR^0,-)$.

下设 $[a,b]$ 是一个区间. 则我们可以用以下两种方式指定相容的定向:

\img{0.6}{12.3.13.png}

设 $S$ 为 $k$ 维分片光滑曲面. 设 $\set{N_i}$ 为至多可数个低维光滑曲面,且 $S$ 在去掉这些低维曲面后可以写成若干个 $k$ 维光滑曲面的不交并:
$$
S\setminus\set{N_i}=\bigsqcup_jS_j
$$

设 $\set{S_j}$ 均可定向且都已经指定了一个定向.

取定 $S_i$ 与 $S_j$. 设
$$
\overline{S}_i\cap\overline{S}_j=\bigsqcup_r\Gamma_r
$$

其中 $\Gamma_r$ 要么是 $k-1$ 维光滑曲面,要么是 $\le k-2$ 维的曲面.

若 $S_i$ 与 $S_j$ 在每个 $k-1$ 维 $\Gamma_r$ 上诱导相反的定向,则称 $S_i$ 与 $S_j$ 的定向相容.

\begin{definition}
    设 $S$ 为 $k$ 维分片光滑曲面.

    称 $S$ 可定向,若在去掉至多可数个低维曲面之后,$S$ 是若干个 $k$ 维光滑曲面 $S_i$ 的并,且 $\forall i\ne j$ 有 $S_i$ 与 $S_j$ 的定向相容.

    此时我们称 $\set{S_i|i\in I}$ 的定向决定了一个 $S$ 的定向.
\end{definition}

\img{0.6}{12.3.14.png}

\begin{hint}
    在分片光滑曲面的定义中,卓里奇《数学分析》并没有明确说明“若干”究竟指的是有限还是可数.

    但是,如果我们采取“可数”作为定义,曲面可定向性的一致性将会被破坏(即习题 12.3.4),也就是:存在光滑曲面 $S$ 不可定向,但 $S$ 在分片光滑曲面的意义下可定向.

    反例如下:考虑一条 Möbius 带

    \img{0.6}{12.3.15.png}

    在其上取一条分界线 $\Gamma$. 在 $\Gamma$ 两侧分别可以取越来越细的可数个曲面 $S_1,S_2,\cdots$ 与 $\widetilde{S}_1,\widetilde{S}_2,\cdots$. 在分界线两侧,取相反的定向.

    则对任意 $i,j$ 有 $\overline{S_i}\cap\overline{\widetilde{S}_j}=\varnothing$. 从而 $S_i$ 与 $\widetilde{S}_j$ 的定向相容.

    故在分片光滑曲面的意义下,Möbius 带是可定向的.

    从这一理由出发,采取“有限”作为定义或许是更加合理的选择.
\end{hint}