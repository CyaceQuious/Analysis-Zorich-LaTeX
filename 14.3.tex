\mysubsection{具有标量势的场}

\begin{definition}
    设 $D\subset\RR^n$ 为开集,$A:D\to\RR^n$ 为 $C^{(1)}$ 光滑.
    
    称 $U:D\to\RR$ 是 $A$ 的势,若 $\nabla U=A$.

    此时称 $A$ 是一个势场.
\end{definition}

\begin{property}
    若 $D$ 为区域,$A:D\to\RR^n$ 为势场,$U$ 与 $U'$ 均为 $A$ 的势,则 $\exists C\in\RR,U'=U+C$.
\end{property}

\begin{example}[ 万有引力场]
    定义 $F:\RR^3\setminus\set{0}\to\RR^3$ 为
$$
F(r)=-\frac{GM}{\abs{r}^2}\frac{r}{\abs{r}}=-\frac{GMr}{\abs{r}^3}
$$

    $F$ 的一个势为
$$
U(r)=\frac{GM}{\abs{r}}
$$
\end{example}

\begin{example}[ 点电荷的静电场]
    定义 $E:\RR^3\setminus\set{0}\to\RR^3$ 为
$$
F(r)=\frac{q}{4\pi\eps_0}\frac{r}{\abs{r}^3}
$$

    $F$ 的一个势为
$$
U(r)=-\frac{q}{4\pi\eps_0\abs{r}}
$$
\end{example}

\begin{property}
    设 $D\subset\RR^n$ 为开集,$A:D\to\RR^n$ 为势场,$U:D\to\RR$ 为 $A$ 的势.
    
    任给 $\gamma\subset D$ 为光滑曲线,有
$$
\int_\gamma A\cdot\dd\mathbf{s}=U\biggr |_{\partial\gamma}=U(q)-U(p)
$$

    其中 $p,q$ 分别为 $\gamma$ 的起点与终点.
\end{property}

\mysubsection{存在标量势的必要条件}

\begin{property}
    设 $D\subset\RR^n$ 为开集. 则若 $A:D\to\RR^n$ 为势场,必有
$$
\dd\omega_A^1=0
$$

    特别的,若 $n=3$,则 $A$ 为势场 $\implies\nabla\times A$=0.
\end{property}
\begin{proof}
    设 $U:D\to\RR$ 为 $A$ 的势. 即 $A=\nabla U$. 从而
$$
\dd\omega_A^1=\dd\dd\omega_U^0=\dd^2\omega_U^0=0
$$

    当 $n=3$ 时,$\dd\omega_A^1=\omega_{\nabla\times A}^2=0\implies\nabla\times A=0$.
\end{proof}

\begin{hint}
    一般来讲,$\dd\omega_A^1=0$ 仅为必要条件,不一定充分.
\end{hint}

\begin{example}
    定义 $A:\RR^2\setminus\set{0}\to\RR^2$ 为
$$
A(x,y)=\left(\frac{-y}{x^2+y^2},\frac{x}{x^2+y^2}\right)
$$
    
    则直接计算可得 $\dd\omega_A^1=0$.

    但我们断言:$A$ 不是势场. 反证,设 $A$ 为势场.

    取 $\gamma$ 为单位圆周,由 $A$ 为势场可得
$$
\int_\gamma\omega_A^1=0
$$

    但另一方面,我们已经计算过
$$
\int_\gamma\omega_A^1=2\pi
$$

    矛盾. 故 $A$ 不是势场.
\end{example}

\mysubsection{存在标量势的充要条件}

以下我们给出一个重要的判定准则来判定向量场何时具有标量势. 为此我们首先给出一个定义.

\begin{definition}
    设 $D\subset\RR^n$ 为开集,$\omega\in\Omega^1(D)$.

    设 $\gamma:[a,b]\to D$ 是一条分段光滑路径,我们定义 $\omega$ 沿 $\gamma$ 的积分为
$$
\int_\gamma\omega\triangleq\int_a^b\gamma^*\omega
$$
\end{definition}

\begin{hint}
    \begin{enumerate}
        \item 当 $\gamma([a,b])$ 为光滑曲线且 $\gamma$ 为其参数化时,上面的定义即为通常的曲线积分的定义.
        
        \item 但以上的定义更广泛,例如我们可以取一些光滑路径,使得其像并非光滑曲线,但此时我们依然可以谈论 $\omega$ 沿 $\gamma$ 的积分.
    \end{enumerate}
\end{hint}

\img{0.8}{14.3.1.png}

现在我们可以来谈论这个充要条件了.

\begin{property}
    设 $D\subset\RR^n$ 为开集,$A:D\to\RR^n$ 为 $C^{(1)}$ 光滑.

    则 $A$ 为势场 $\iff$ 对任意 $D$ 中的分段光滑闭路径 $\gamma$ 有
$$
\oint_\gamma A\cdot\dd\mathbf{s}\triangleq\int_\gamma\omega_A^1=0
$$
\end{property}
\begin{proof}
    $\implies$:设 $U\to\RR$ 满足 $A=\nabla U$.
    
    设闭路径 $\gamma:[a,b]\to D$. 则 $\gamma(a)=\gamma(b)$. 从而
$$
\begin{aligned}
    \int_\gamma\omega_A^1&=\int_\gamma\omega_{\nabla U}^1=\int_\gamma\sum_{i=1}^n\pard{U}{x_i}(x)\dd x_i\\
    &=\int_a^b(U(\gamma(t)))'\dd t=U(\gamma(t))\biggr |_a^b=0
\end{aligned}
$$

    $\impliedby$:不妨设 $D$ 为区域(即联通开集).

    取定 $x_0\in D$. 我们首先断言:任取 $x\in D$,设 $\gamma_1$ 与 $\gamma_2$ 是连接 $x_0$ 与 $x$ 的两个分段光滑路径,则
$$
\int_{\gamma_1}\omega_A^1=\int_{\gamma_2}\omega_A^1
$$

    证明:设 $\gamma_1:[0,1]\to D,\gamma_2:[0,1]\to D$ 为两条分段光滑路径,$\gamma_1(0)=\gamma_2(0)=x_0$ 且 $\gamma_1(1)=\gamma_2(1)=x$.

    定义 $\gamma:[0,2]\to D$ 为
$$
\gamma(t)=\begin{cases}
    \gamma_1(t) & t\in[0,1]\\
    \gamma_2(2-t) & t\in(1,2]
\end{cases}
$$

    \img{0.7}{14.3.2.png}

    则 $\gamma$ 为分段光滑闭路径. 从而
$$
\int_\gamma\omega_A^1=\int_{\gamma_1}\omega_A^1-\int_{\gamma_2}\omega_A^2=0
$$

    \qed

    由此,我们可以定义 $U:D\to\RR$ 为
$$
U(x)\triangleq\int_\gamma\omega_A^1
$$

    其中 $\gamma$ 为连接 $x_0$ 与 $x$ 的任意一条分段光滑路径.

    下证:$U$ 即为 $A$ 的势.

    取定 $x\in D$. 设 $r>0$ 使得 $B(x,r)\subset D$.

    对任意 $0<\eps<r,1\le i\le n$ 定义 $\gamma_\eps:[0,\eps]\to D$ 为 $\gamma_\eps(t)\triangleq x+te_i$. 则有
$$
\begin{aligned}
    U(x+\eps e_i)-U(x)&=\int_{\gamma_\eps}\omega_A^1\\
    &=\int_{\gamma_\eps}\sum_{j=1}^nA_j(x)\dd x_j\\
    &=\int_0^\eps A_i(x+te_i)\dd t\\
    &=\eps A_i(x+\eta e_i)
\end{aligned}
$$

    其中 $\eta\in[0,\eps]$.

    \img{0.3}{14.3.3.png}

    由 $A$ 为 $C^{(1)}$ 光滑可得
$$
\lim_{\eps\to 0}\frac{U(x+\eps e_i)-U(x)}{\eps}=A_i(x)
$$

    故 $\nabla U=A$. 故 $U$ 即为所求.
\end{proof}

\begin{hint}
    从证明的过程来看,我们完全可以考虑一类更小更简单的闭路径,即那些由有限条与坐标轴平行的线段构成的折线构成的路径类. 这个观察可以简化下一个性质的证明.
\end{hint}
