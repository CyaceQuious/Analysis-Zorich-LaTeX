在学完一元积分理论后,作为应用我们知道了如何求光滑曲线的长度,以及光滑旋转体的体积.
在学完多元积分理论后,我们知道了如何求 Jordan 可测集的体积.
但目前为止,我们还不知道如何求曲面的面积. 本节我们来讨论这个问题.

\mysubsection{平行多面体的体积}

设 $\xi_1,\cdots,\xi_k$ 是 $\RR^k$ 中的 $k$ 个向量. 则由线性代数知识可知:

由 $\xi_1,\cdots,\xi_k$ 张成的平行多面体的(有向)体积由行列式
$$
\det\xi=\det[\xi_1,\cdots,\xi_k]
$$

给出.

若 $\det\xi\ne 0$ ,其符号恰好对应于标架 $\xi$ 的两种可能定向.

如果我们只关心体积的大小,则其值由 $\abs{\det\xi}$ 给出.

事实上,有一种更对称的方式来表示这个值. 设方阵 $G\triangleq\xi^T\xi$ ,则
$$
\abs{\det\xi}=\sqrt{\det\xi^T\xi}=\sqrt{\det G}
$$

注意到 $G=[\inner{\xi_i,\xi_j}]_{i,j}$.

更为重要的是,这一结论还可以推广到一般情形:

\begin{property}
    设 $k\le n,\xi_1,\cdots,\xi_k\in\RR^n$.

    则由 $\xi_1,\cdots,\xi_k$ 张成的平行多面体的体积为
$$
V(\xi_1,\cdots,\xi_k)=\sqrt{\det\xi^T\xi}=\sqrt{\det[\inner{\xi_i,\xi_j}]_{i,j}}
$$
\end{property}

\mysubsection{光滑参数化曲面面积的定义}

设 $S\subset\RR^n$ 为 $k$ 维光滑曲面,且存在参数化 $\varphi:D\to S$.

其中 $D\subset\RR^k$ 为开集,$\varphi\in C^{(1)}(D)$ 且为同胚.

\img{0.8}{12.4.1.png}

设 $t\in D$. 考虑在 $t$ 处由 $h_1e_1,\cdots,h_ke_k$ 张成的 $k$ 维矩形 $I$. 其在映射 $\varphi$ 下变为 $S$ 内的一个曲边多面体.

由微分的几何意义知:当 $h_i$ 很小时,可以用在 $\varphi(t)$ 处的由切向量 $h_1\varphi'(t)e_1,\cdots,h_k\varphi'(t)e_k$ 张成的平行多面体近似替换该曲边多面体.

从而
$$
\begin{aligned}
    V(\varphi(I))\approx &V(h_1\varphi'(t)e_1,\cdots,h_k\varphi'(t)e_k)\\
    =&h_1\cdots h_k\det\sqrt{[\inner{\partial_i\varphi(t),\partial_j\varphi(t)}]_{i,j}}\\
    =&\det\sqrt{G(t)}V(I)
\end{aligned}
$$

其中 $G(t)\triangleq\varphi'(t)^T\varphi'(t)=[\inner{\partial_i\varphi(t),\partial_j\varphi(t)}]_{i,j}$.

现在设想我们将 $\RR^k$ 分成很小的矩形 $P=\set{I_\alpha}$ ,并考虑那些完全位于 $D$ 中的矩形 $I_\alpha$.

我们用 $\sqrt{\det G(t_\alpha)}$(其中 $t_\alpha$ 为 $I_\alpha$ 的端点)来近似地逼近 $\varphi(I_\alpha)$ 的体积. 从而
$$
V(S)\approx\sum_{I_\alpha\subset D}V(\varphi(I_\alpha))\approx\sum_{I_\alpha\subset D}\sqrt{\det G(t_\alpha)}V(I_\alpha)
$$

当 $P$ 的步长趋近于 $0$ 时
$$
\lim_{\lambda(P)\to 0}\sum_{I_\alpha\subset D}\sqrt{\det G(t_\alpha)}V(I_\alpha)=\int_D\sqrt{\det G(t)}\dd t
$$

基于以上的先验推理,如下的定义就变得很合理了:

\begin{definition}\label{df:area}
    设 $S$ 为 $k$ 维光滑曲面,且有参数化 $\varphi:D\to S$. 定义 $S$ 的面积($k$ 维体积)为
$$
V_k(S)\triangleq\int_D\sqrt{\det G(t)}\dd t=\int_D\sqrt{\det[\inner{\partial_i\varphi(t),\partial_j\varphi(t)}]_{i,j}}\dd t
$$

    若右式的极限存在.
\end{definition}

\begin{hint}
    \begin{enumerate}
        \item 可以认为右边的积分是开集 $D$ 上的广义积分.
        
        注意到积分函数 $\sqrt{\det G(t)}$ 连续且非负,从而以上的积分要么为有限非负值,要么为 $+\infty$.

        当极限为 $+\infty$ 时,我们称 $S$ 的面积为无穷大.

        \item 当 $k=1$ 时,$S$ 为一条参数化曲线. 设 $D=[a,b]$ ,则以上定义简化为
$$
V_1(S)=\int_a^b\sqrt{\inner{\varphi'(t),\varphi'(t)}}\dd t=\int_a^b\abs{\varphi'(t)}\dd t
$$
        
        这一定义与我们之前通过物理意义给出的定义是相同的. 现在,我们得以从几何的角度再次得到这一定义.

        \item 当 $k=n$ 时
$$
V_n(S)=\int_D\sqrt{\det \varphi'(t)^T\varphi'(t)}\dd t=\int_D\abs{\det\varphi'(t)}\dd t\xlongequal{u=\varphi(t)}\int_S1\dd u=\mu(S)
$$

        从而此时的体积 $V_n(S)$ 即为 $S$ 的 Jordan 测度 $\mu(S)$.

        \item 当 $k=2,n=3$ 时,传统上我们记
$$
G(t)=\begin{bmatrix}
    E & F\\
    F & G
\end{bmatrix}\implies
E=\inner{\partial_1\varphi(t),\partial_1\varphi(t)},
F=\inner{\partial_1\varphi(t),\partial_2\varphi(t)},
G=\inner{\partial_2\varphi(t),\partial_2\varphi(t)}
$$

        此时,三维空间中的二维曲面面积为
        \begin{property}
$$
V_2(S)=\iint_D\sqrt{EG-F^2}\dd t_1\dd t_2
$$
        \end{property}

        当 $S$ 是 $f:D\to\RR$ 的图像时有
$$
\begin{aligned}
    \varphi(t_1,t_2)&=(t_1,t_2,f(t_1,t_2))\\
    \varphi'(t)&=\begin{bmatrix}
        1 & 0\\
        0 & 1\\
        f'_x(t) & f'_y(t)
    \end{bmatrix}\implies
    E=1+(f'_x)^2,
    F=f'_xf'_y,
    G=1+(f'_y)^2\\
    EG-F^2&=1+(f'_x)^2+(f'_y)^2
\end{aligned}
$$

        从而我们得到
        \begin{property}
$$
V_2(S)=\iint_D\sqrt{1+(f'_x)^2+(f'_y)^2}\dd x\dd y
$$
        \end{property}

        \item 注意到我们对曲面面积的定义是在给定的参数化 $\varphi$ 下定义的. 一个自然的问题是:该定义是否不依赖于参数化的选取?答案是肯定的.
    \end{enumerate}
\end{hint}

\begin{property}
    可参数化曲面面积的定义不依赖于参数化的选取.
\end{property}
\begin{proof}
    设 $S\subset\RR^n$ 是一个 $k$ 维可参数化光滑曲面.

    设 $\varphi:D\to S,\psi:\widetilde{D}\to S$ 是 $S$ 的两个参数化. 我们有
$$
\begin{aligned}
    V_k(S)&=\int_D\sqrt{\det{G(t)}}\dd t\\
    \widetilde{V}_k(S)&=\int_{\widetilde{D}}\sqrt{\det{\widetilde{G}(t)}}\dd t
\end{aligned}
$$

    注意到 $\Phi=\psi^{-1}\circ\varphi:D\to\widetilde{D}$ 为微分同胚.

    \img{0.6}{12.4.2.png}

    由 $\varphi=\psi\circ\Phi$ 知 $\varphi'(t)=\psi'(\Phi(t))\Phi'(t)$. 从而
$$
\begin{aligned}
    G(t)&=\varphi'(t)^T\varphi'(t)\\
    &=\Phi'(t)^T\psi'(\Phi(t))^T\psi'(\Phi(t))\Phi'(t)\\
    &=\Phi'(t)^T\widetilde{G}(\Phi(t))\Phi'(t)
\end{aligned}
$$

从而由变量替换公式可得
$$
\begin{aligned}
\widetilde{V}_k(S)&=\int_{\widetilde{D}}\sqrt{\det\widetilde{G}(t)}\dd t\\
&=\int_D\sqrt{\det\widetilde{G}(\Phi(t))}\abs{\det\Phi'(t)}\dd t\\
&=\int_D\sqrt{\det\widetilde{G}(\Phi(t))}\sqrt{\det\Phi'(t)^T\Phi'(t)}\dd t\\
&=\int_D\sqrt{\det G(t)}\dd t\\
&=V_k(S)
\end{aligned}
$$
\end{proof}

\mysubsection{光滑曲面面积的定义}

接下来我们讨论如何定义一般光滑曲面的面积.
受可参数化曲面面积定义的启发,此时最自然的方法是:将 $S$ 分割成一些互不相交的片,使得每个片可参数化,然后再将它们的面积相加.
但在技术上,这涉及到是否与分割方式有关,是否带边等问题. 所以我们先来做一些准备.

\mysubsubsection{Lebesgue 意义下的零面积集}

\begin{definition}
    设 $N\subset\RR^n$. 称 $N$ 为 Lebesgue 意义下的 $k$ 维零面积集,若 $\forall\eps>0$ ,存在至多可数个 $k$ 维可参数化光滑曲面 $\set{S_i}$ 使得
    \begin{enumerate}
        \item $\displaystyle N\subset\bigcup_i S_i$
        \item $\displaystyle \sum_{i}V_k(S_i)<\eps$
    \end{enumerate}

    我们将其简称为 L--零 $k$ 维面积集.
\end{definition}

直观来说,这样的集合在 $k$ 维面积的意义下是可忽略的.

我们也使用记号 $\text{L--}V_k(N)=0$.

与之前讨论过的 Lebesgue 零测集的性质类似,我们也有

\begin{property}
    \begin{enumerate}
        \item 若 $\forall i,\text{L--}V_k(N_i)=0$ ,则 $\displaystyle\text{L--}V_k\left(\bigcup_iN_i\right)=0$.
        
        \item 若 $\text{L--}V_k(N)=0$ ,则对 $N'\subset N$ 有 $\text{L--}V_k(N')=0$.
        
        \item 若 $N$ 为至多可数个 $\le k-1$ 维光滑曲面的并,则 $\text{L--}V_k(N)=0$,
    \end{enumerate}
\end{property}

特别的,对于一个 $k-1$ 维分片光滑曲面 $S$ 有 $\text{L--}V_k(S)=0$.

\mysubsubsection{一般光滑曲面面积的定义}

如下的观察对于定义面积很重要:

\begin{lemma}
    设 $S$ 为 $k$ 维光滑可参数化曲面,$\varphi:D\to S$ 是一个参数化.

    若 $\widetilde{S}\subset S$ 是一个 $l$ 维分片光滑曲面,$l<k$ ,则 $\varphi^{-1}(\widetilde{S})\subset D$ 是一个 $\RR^k$ 中的 Lebesgue 零测集.
\end{lemma}
\begin{proof}
    首先注意到 $\widetilde{S}=\bigcup_i\widetilde{S}_i$ ,其中每个 $\widetilde{S}_i$ 均为光滑曲面且维数 $l_i\le l$.

    再注意到,对每个 $\widetilde{S}_i$ ,可以将 $\varphi^{-1}(\widetilde{S}_i)$ 表示为一个 $l_i$ 维区间的图像(光滑映射).

    从而有 $\varphi^{-1}(\widetilde{S}_i)$ 为 Lebesgue 零测集. 具体细节留作习题.
\end{proof}

\img{0.5}{12.4.3.png}

现在我们可以给出一般曲面面积的定义了.

\begin{definition}
    设 $S\subset\RR^n$ 为 $k$ 维光滑曲面.

    若在去掉至多可数个 $\le k-1$ 维的光滑曲面 $\set{N_i|i\in I}$ 后有
    $$
    S\setminus\bigsqcup_iN_i=\bigsqcup_jS_j
    $$

    其中 $\set{S_j}$ 为至多可数个 $k$ 维光滑可参数化曲面. 则定义
    $$
    V_k(S)\triangleq\sum_jV_k(S_j)
    $$
\end{definition}
\begin{hint}
    当然,为了使这个定义有意义,我们需要说明:
\end{hint}

\begin{property}
    以上定义中的分解 $\set{N_i},\set{S_j}$ 存在.
\end{property}

另一方面,我们需要验证曲面面积的定义不依赖于以上分解的选取.

为此,我们首先需要说明:将可参数化曲面进一步划分保持总面积不变.

\begin{lemma}
    设 $S$ 为 $k$ 维可参数化曲面,且存在至多可数个 $\le k-1$ 维光滑曲面 $\set{N_i}$ 使得
$$
S\setminus\bigsqcup_i N_i=\bigsqcup_jS_j
$$

    其中 $\set{S_j}$ 为至多可数个 $k$ 维可参数化光滑曲面. 则有
$$
V_k(S)=\sum_jV_k(S_j)
$$
\end{lemma}
\begin{proof}
    设 $\varphi:D\to S$ 为 $S$ 的参数化.

    记 $D_j=\varphi^{-1}(S_j)$. 则 $\varphi:D_j\to S_j$ 为 $S_j$ 的参数化.

    \img{0.5}{12.4.4.png}

    记 $N=\bigcup\limits_iN_i$. 则我们已经知道
$$
\varphi^{-1}(N)=D\setminus\bigsqcup_jD_j
$$

    为 $\RR^k$ 中的 Lebesgue 零测集. 从而

$$
\begin{aligned}
    V_k(S)&=\int_D\sqrt{\det G(t)}\dd t=\int_{D\setminus\varphi^{-1}(N)}\sqrt{\det G(t)}\dd t\\
    &=\int_{\bigcup\limits_jD_j}\sqrt{\det G(t)}\dd t=\sum_j\int_{D_j}\sqrt{\det G(t)}\dd t\\
    &=\sum_jV_k(S_j)
\end{aligned}
$$
\end{proof}

现在我们可以验证

\begin{property}
    $S$ 面积的定义与分解的方式无关.
\end{property}
\begin{proof}
    设 $\set{N_i},\set{S_j}$ 与 $\set{\widetilde{N}_i},\set{\widetilde{S}_j}$ 是 $S$ 的两个分解.

    令 $\widetilde{N}=\bigcup\limits_i\widetilde{N}_i$. 我们有
$$
S=\widetilde{N}\sqcup\left(\bigsqcup_{j'}\widetilde{S}_{j'}\right)
$$

    于是
$$
S_j=(S_j\cap\widetilde{N})\sqcup\left(\bigsqcup_{j'}(S_j\cap\widetilde{S}_{j'})\right)
$$

    应用前面的引理,可以得到
$$
V_k(S_j)=\sum_{j'}V_k(S_j\cap\widetilde{S}_{j'})
$$

    \img{0.8}{12.4.5.png}

    进而有
$$
V_k(S)=\sum_jV_k(S_j)=\sum_j\sum_{j'}V_k(S_j\cap\widetilde{S}_{j'})
$$

    对称地有
$$
\widetilde{V}_k(S)=\sum_{j'}V_k(\widetilde{S}_{j'})=\sum_{j'}\sum_jV_k(\widetilde{S}_{j'}\cap S_j)
$$

    即证 $V_k(S)=\widetilde{V}_k(S)$.
\end{proof}

至此,我们完成了光滑曲面面积的定义.

\mysubsubsection{分片光滑曲面面积的定义}

\begin{definition}
    设 $S$ 为 $k$ 维分片光滑曲面. 已知存在至多可数个 $\le k-1$ 维分片光滑曲面 $\set{N_i}$ 使得
$$
S\setminus\bigsqcup_iN_i=\bigsqcup_jS_j
$$

    其中 $\set{S_j}$ 为至多可数个 $k$ 维光滑曲面. 则我们定义 $S$ 的面积为
$$
V_k(S)\triangleq\sum_jV_k(S_j)
$$
\end{definition}

\begin{property}
    以上的定义不依赖于分解 $\set{N_i},\set{S_j}$ 的选取.
\end{property}