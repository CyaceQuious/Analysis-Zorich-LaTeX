\mysubsection{标量场与向量场}

场的概念十分简单但又十分重要.

\begin{definition}
    设 $U\subset\RR^n$ 为开集,$V$ 为 Banach 空间.

    设映射 $F:U\to V$,则称 $F$ 是定义在 $D$ 上的场.

    若 $F$ 连续,则称 $F$ 为连续场. 若 $F$ 光滑,则称 $F$ 为光滑场.

    \begin{enumerate}
        \item $V=\RR$. 则称 $F$ 为标量场.
        
        \item $V=\RR^n$. 则称 $F$ 为向量场.
        
        \item $V=\mathscr{A}^p(\RR^n)$. 则称 $F$ 为形式场.
    \end{enumerate}
\end{definition}

\mysubsection{$\RR^3$ 中的标量场,向量场与形式场}

当空间的维数为 $3$ 时,向量场与形式场会呈现某种奇妙的关系. 下面我们来解释这种关系.

\mysubsubsection{代数层面的对应}

\begin{property}
    在 $\RR^3$ 中取定标准内积与标准定向.

    \begin{enumerate}
        \item 对任意 $\omega\in\mathscr{A}^1(\RR^3)$,存在唯一的 $A\in\RR^3$ 使得 $\omega(\xi)=\inner{A,\xi},\forall\xi\in\RR^3$.
        
        \item 对任意 $\omega\in\mathscr{A}^2(\RR^3)$,存在唯一的 $B\in\RR^3$ 使得 $\omega(\xi_1,\xi_2)=\det(B,\xi_1,\xi_2),\forall\xi_1,\xi_2\in\RR^3$.
    \end{enumerate}
\end{property}