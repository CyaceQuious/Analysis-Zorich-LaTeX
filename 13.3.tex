本节,我们证明三个重要的积分公式. 从证明过程可以看出它们都是 Newton-Leibniz 公式的推论.

\mysubsection{Green 公式}

\begin{theorem}[Green 公式]
    设 $D\subset\RR^2$ 为区域,$\overline{D}$ 紧,$\partial D=\partial\overline{D}$ 为分片光滑曲线.

    设 $P,Q:\overline{D}\to\RR$ 为 $C^{(1)}$ 光滑. 则
$$
\int_D\left(\pard{Q}{x}-\pard{P}{y}\right)\dd x\dd y=\int_{\partial D}P\dd x+Q\dd y
$$

    其中 $D$ 取标准定向 $(e_1,e_2)$,而 $\partial D$ 上取由 $\overline{D}$ 的定向所诱导的定向.
\end{theorem}

我们仅对一类特殊的区域 $D$ 证明上式. 从实用的角度来说,这一类区域已经够用了. 以下我们给出定义

\begin{definition}
    称 $D\subset\RR^2$ 是一个 $x$-型区域,若存在 $\varphi_1,\varphi_2:[a,b]\to\RR$ 连续且分段光滑,满足 $\varphi_1(x)\le\varphi_2(x),\forall x\in [a,b]$,使得
$$
\overline{D}=\set{(x,y)\in\RR^2|x\in[a,b],\varphi_1(x)\le y\le\varphi_2(x)}
$$
\end{definition}

\img{0.8}{13.3.1.png}

类似地可以定义 $y$-型区域. 我们省略其定义.

\begin{lemma}
    设 $D$ 为 $x$-型区域,且由 $\varphi_1,\varphi_2:[a,b]\to\RR$ 给定. 设 $P\in C^{(1)}(\overline{D})$. 则
$$
\int_{\partial D}P\dd x=\int_D-\pard{P}{y}\dd x\dd y
$$
\end{lemma}
\begin{proof}
    一方面,由 Fubini 定理有
$$
\begin{aligned}
    \int_D-\pard{P}{y}\dd x\dd y&=-\int_a^b\int_{\varphi_1(x)}^{\varphi_2(x)}\pard{P}{y}(x,y)\dd y\dd x\\
    &\xlongequal{\text{N-L}}-\int_a^b\left(P(x,\varphi_2(x))-P(x,\varphi_1(x))\right)\dd x\\
    &=\int_a^b P(x,\varphi_1(x))\dd x-\int_a^b P(x,\varphi_2(x))\dd x
\end{aligned}
$$

\img{0.5}{13.3.2.png}

    另一方面,$\partial D=\gamma_1\cup\gamma_2\cup\gamma_3\cup\gamma_4$ 如图. 其由 $\overline{D}$ 诱导的定向为逆时针方向. 我们分别给出它们的参数化
$$
\begin{aligned}
    &\gamma_1:\quad\phi_1:[a,b]\to\RR^2,t\mapsto(t,\varphi_1(t))\\
    &\gamma_2:\quad\phi_2:[\varphi_1(b),\varphi_2(b)]\to\RR^2,t\mapsto(b,t)\\
    &\gamma_3:\quad\phi_3:[a,b]\to\RR^2,t\mapsto(t,\varphi_2(t))\\
    &\gamma_4:\quad\phi_4:[\varphi_1(a),\varphi_2(a)]\to\RR^2,t\mapsto(a,t)\\
\end{aligned}
$$

    其中 $\phi_1,\phi_2$ 诱导了 $\gamma_1,\gamma_2$ 上的定向,而 $\phi_3,\phi_4$ 诱导了 $\gamma_3,\gamma_4$ 上相反的定向. 从而
$$
\begin{aligned}
    &\int_{\gamma_1}P\dd x=\int_a^bP(t,\varphi_1(t))\dd t\\
    &\int_{\gamma_2}P\dd x=0\\
    &\int_{\gamma_3}P\dd x=-\int_a^bP(t,\varphi_2(t))\dd t\\
    &\int_{\gamma_4}P\dd x=0
\end{aligned}
$$

    综上
$$
\int_{\partial D}P\dd x=\int_D-\pard{P}{y}\dd x\dd y
$$
\end{proof}

\begin{lemma}\label{green:x}
    设 $D\subset\RR^2$ 为区域,存在有限个不交的 $x$-型区域 $D_1,\cdots,D_k$ 使得 
$$
\overline{D}=\overline{D}_1\cup\cdots\cup\overline{D}_k
$$

    设 $P\in C^{(1)}(\overline{D})$,则
$$
\int_{\partial D}P\dd x=\int_D-\pard{P}{y}\dd x\dd y
$$
\end{lemma}
\begin{proof}
    一方面
$$
\begin{aligned}
    \int_D-\pard{P}{y}\dd x\dd y&=\sum_{i=1}^k-\int_{D_i}\pard{P}{y}\dd x\dd y\\
    &=\sum_{i=1}^k\int_{\partial D_i}P\dd x
\end{aligned}
$$

    另一方面,已知若 $D_i$ 与 $D_j$ 有公共边界,则 $D_i$ 与 $D_j$ 在公共边界上诱导相反的定向. 故
$$
\sum_{i=1}^k\int_{\partial D}P\dd x=\int_{\partial D}P\dd x
$$
\end{proof}

\img{0.8}{13.3.3.png}

同理可以证明

\begin{lemma}\label{green:y}
    设 $D\subset\RR^2$ 为区域,存在有限个不交的 $y$-型区域 $G_1,\cdots,G_k$ 使得 
$$
\overline{D}=\overline{G}_1\cup\cdots\cup\overline{G}_k
$$

    设 $Q\in C^{(1)}(\overline{D})$,则
$$
\int_{\partial D}Q\dd y=\int_D\pard{Q}{x}\dd x\dd y
$$
\end{lemma}

因此我们做出如下的定义

\begin{definition}
    称 $D$ 是一个初等区域,若存在有限个 $x$-型区域 $D_1,\cdots,D_k$ 满足
$$
D_i\cap D_j=\varnothing,\forall i\ne j~\text{且}~\overline{D}=\bigcup_{i=1}^k\overline{D}_i
$$

    且存在有限个 $y$-型区域 $G_1,\cdots,G_k$ 满足
$$
    G_i\cap G_j=\varnothing,\forall i\ne j~\text{且}~\overline{D}=\bigcup_{i=1}^k\overline{G}_i
$$
\end{definition}

\img{0.8}{13.3.4.png}

将引理 \ref{green:x} 与引理 \ref{green:y} 相加,我们即对初等区域证明了 Green 公式.

\begin{hint}
    \begin{enumerate}
        \item 我们可以这样来看 Green 公式:
        
        令 $\omega=P\dd x+Q\dd y$ 为 $\overline{D}$ 上的 $1$-形式. 则
$$
\dd\omega=-\pard{P}{y}\dd x\wedge\dd y+\pard{Q}{x}\dd x\wedge\dd y
$$

        从而 Green 公式变为
$$
\int_{\partial D}\omega=\int_D\dd\omega
$$
    \end{enumerate}
\end{hint}